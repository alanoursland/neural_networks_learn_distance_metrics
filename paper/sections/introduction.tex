\section{Introduction}
\label{sec:introduction}

Neural networks have transformed machine learning through their ability to learn complex, hierarchical representations, traditionally interpreted through intensity-based representations where larger activation values signify stronger feature presence. This interpretation, originating from the McCulloch-Pitts neuron \cite{mcculloch1943logical} and Rosenblatt's perceptron \cite{rosenblatt1958perceptron}, underlies modern deep learning. However, our theoretical understanding of neural networks' internal mechanisms remains limited \cite{lipton2018mythos}, particularly regarding how they represent and process features.

Recent theoretical work challenges this intensity-based paradigm, suggesting networks may naturally learn distance-based representations \cite{oursland2024interpreting} where smaller activations indicate proximity to learned prototypes. This reinterpretation not only revisits long-held assumptions but also provides a statistical foundation rooted in principles like the Mahalanobis distance \cite{mahalanobis1936generalized}. Empirical evidence supports the notion that distance-based metrics play a crucial role in how networks learn and utilize features \cite{oursland2024neural}, suggesting the need to rethink the fundamental nature of representations.

\textbf{Core Questions.} This paper investigates: (1) whether neural networks naturally prefer distance-based or intensity-based representations, (2) how architectural choices shape these representational biases, and (3) what geometric and statistical principles underlie these preferences.

\textbf{Contributions.} We examine neural network behavior through distance and intensity representations via:
\begin{enumerate}
    \item A theoretical framework formalizing the distinction between these representations
    \item Empirical analysis of six architectural variants, revealing mechanisms behind dead node creation and geometric performance limitations
    \item Introduction of OffsetL2, a novel architecture validating our framework through strong, stable performance
\end{enumerate}

The remainder of this paper reviews related work (Section \ref{sec:related_work}), establishes theoretical foundations (Section \ref{sec:background}), presents our experimental design (Section \ref{sec:exp_design}) and findings (Section \ref{sec:results}), and discusses implications (Section \ref{sec:discussion}).