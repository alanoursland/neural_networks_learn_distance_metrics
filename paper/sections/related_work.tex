\section{Related Work}
\label{sec:related_work}
Our work builds upon recent advances in understanding neural networks through statistical distance metrics. Prior research has demonstrated how linear layers with Absolute Value (Abs) activations approximate the Mahalanobis distance \cite{mahalanobis1936generalized, oursland2024interpreting}, providing a mathematical foundation for distance-based representations. Empirical studies have shown that networks with ReLU and Abs activations exhibit particular sensitivity to perturbations affecting distance relationships in the feature space \cite{oursland2024neural}, suggesting distance metrics play a fundamental role in how networks process information.

Alternative approaches to incorporating distance metrics in neural networks include Radial Basis Function (RBF) networks \cite{broomhead1988radial}, which use distances from learned centers for classification, Siamese networks \cite{bromley1994signature}, which learn embeddings where distances represent similarity, and Learning Vector Quantization (LVQ) \cite{kohonen1995learning}, which explicitly models class prototypes and uses distance-based classification. While these methods demonstrate the effectiveness of distance-based learning for specific tasks, they have not seen widespread adoption in general-purpose deep learning architectures.

Complementing these distance-centric views, geometric interpretations of neural computation offer valuable insights for understanding internal representations \cite{montavon2018methods, olah2017feature, samek2019explainable}. These approaches analyze hyperplanes and decision boundaries to explain how networks partition and represent data \cite{lipton2018mythos, erhan2009visualizing}, though they typically focus on networks trained under standard intensity-based assumptions. This work bridges the gap between distance-based and geometric interpretations by investigating how architectural choices influence the emergence of distance-based representations.
